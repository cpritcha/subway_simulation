\documentclass[12pt]{article}
\usepackage[explicit]{titlesec}
\usepackage[margin=0.75in]{geometry}
\usepackage{fancyhdr}
\usepackage{fontspec}
\usepackage{scrextend}
\usepackage{listings}
\usepackage{graphicx}
\usepackage{graphics}
\usepackage{amssymb}
\usepackage{longtable}
\usepackage{hyperref}
\usepackage[section]{placeins}
\usepackage[dvipsnames,svgnames]{xcolor}
\usepackage{ulem}
\usepackage{pdfpages}
\usepackage{cite}
\usepackage{amsmath}
\usepackage{mathtools}

\setmainfont{Cambria}
%\doublespacing
\pagestyle{fancy}
\rfoot{\thepage}
\cfoot{\today}
\lhead{}
\rhead{}
\fancyhead[CO]{\textit{CSE 561: Modeling \& Simulation Theory and Applications}}
\fancyhead[CE]{\textit{Jacob Boydston and Calvin Pritchard}}

\titleformat{\section}
{\normalfont\Large\bfseries}{}{0em}{\colorbox[RGB]{79,129,189}{\parbox{\dimexpr\textwidth-2\fboxsep\relax}{\textcolor{white}{\thesection\quad\MakeUppercase{#1}}}}}

\titleformat{name=\section,numberless}
{\normalfont\Large\bfseries}{}{0em}{\colorbox[RGB]{79,129,189}{\parbox{\dimexpr\textwidth-2\fboxsep\relax}{\textcolor{white}{\MakeUppercase{#1}}}}}

\titleformat{\subsection}
{\normalfont\Large\bfseries}{}{0em}{\colorbox[RGB]{219,229,241}{\parbox{\dimexpr\textwidth-2\fboxsep\relax}{\textcolor{black}{\thesubsection\quad\MakeUppercase{#1}}}}}

\begin{document}
	
\author{
	Jacob Boydston\\Online
	\and
	Calvin Pritchard\\In-Person
}
\title{Modeling of a Subterranean Rail Transport Network}
\date{}

\maketitle
\thispagestyle{fancy}

\begin{table*}[htb]
\centering
\begin{tabular}{|l|c|c|}

\hline
\textbf{List of Tasks} & \textbf{Jacob \% Effort} & \textbf{Calvin \% Effort} \\
\hline
1. Introduction & 20 & 80 \\
\hline
2. Problem Description & 40 & 60 \\
\hline
3. Approach & 50 & 50 \\
\hline
\quad 3.1 Model Development & 50 & 50 \\
\hline
\quad 3.2 Experimental Setups \& Experiments & 50 & 50 \\
\hline
4. Results & 60 & 40 \\
\hline
5. Conclusions & 80 & 20 \\
\hline
6. References & 50 & 50 \\
\hline
Appendices & 50 & 50 \\
\hline

\end{tabular}
\end{table*}

%\tableofcontents

%\begin{abstract}
%	Enter abstract text here.
%\end{abstract}

%\section{Introduction}

Subways are a key form of transportation for people living in cities. This
report investigates the role of subway passing tracks in ensuring that subway
service is not adversely impacted by breakdowns.

Subway delay data from the Toronto Transit Commission (TTC) is used as a guide
to determine the frequency and severity of delays for subways, how far trains 
are from one another when delayed and the delay cause.

Different simulation techniques for subway system scheduling and breakdown are
explored in the background section. Petri nets and collision detection methods
are compared for their ability to simulate subway breakdowns matching the
important properties found in TTC data and for ease of use. 

A Petri net method explored is explored for simulating train movement in the
modeling section. The model is placed within the parallel DEVS formalism to
clarify how the trains, stations and tracks interact with one another over time.
Implementation of the simulation model in \code{DEVS-Suite} is also discussed.

Experiments are created to match scenarios with moderate delays (excluding
flooding or other subway wide delays) found in the TTC data. Track
configurations with different prevalences and placements of passing tracks are
tested to see what impact configurations has on subway delays. All track
configurations tested use a looping track layout without intersections.

Model results for track different track configurations are used to determine how
important the prevalence and placement of passing tracks is in preventing
delays. 
	
\end{document}