\section{Introduction}

Subways are a key form of transportation for people living in cities. This
report investigates the effects of number of subway trains and random delays in the overall passenger capacity and delay characteristics for a subway line. Subway delay data from the Toronto Transit Commission (TTC) is used as a guide to determine the frequency and severity of delays for subways, how far trains are from one another when delayed and the delay cause. Different simulation techniques for subway system scheduling and breakdown are
explored in the background section. Petri nets and collision detection methods
are compared for their ability to simulate subway breakdowns matching the
important properties found in TTC data and for ease of use. A Petri net method is explored for simulating train movement in the modeling section. The model is placed within the parallel DEVS formalism to clarify how the trains, stations and tracks interact with one another over time. Implementation of the simulation model in \code{DEVS-Suite} is also discussed.

Experiments are created to determine the effects the number of trains and the length of delays can have on a subway line over a two hour period. The track
configuration tested use a looping track layout without intersections. While only a single subway line is explored, the simulation is constructed such that multiple subway lines could be simulated concurrently within the same experimental frame. With further development on capabilities already built into the simulation, in particular full integration of the semaphores in track sections to determine train occupation, features such as trains bypassing one another and fully coupled simulations of multiple loops sharing stations and track sections are possible.