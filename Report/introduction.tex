\section{Introduction}

Subways are a key form of transportation for people living in cities. This
report investigates the role of subway passing tracks in ensuring that subway
service is not adversely impacted by breakdowns.

Subway delay data from the Toronto Transit Commission (TTC) is used as a guide
to determine the frequency and severity of delays for subways, how far trains
are from one another when delayed and the delay cause.

Different simulation techniques for subway system scheduling and breakdown are
explored in the background section. Petri nets and collision detection methods
are compared for their ability to simulate subway breakdowns matching the
important properties found in TTC data and for ease of use. 

A Petri net method explored is explored for simulating train movement in the
modeling section. The model is placed within the parallel DEVS formalism to
clarify how the trains, stations and tracks interact with one another over time.
Implementation of the simulation model in \code{DEVS-Suite} is also discussed.

Experiments are created to match scenarios with moderate delays (excluding
flooding or other subway wide delays) found in the TTC data. Track
configurations with different prevalences and placements of passing tracks are
tested to see what impact configurations has on subway delays. All track
configurations tested use a looping track layout without intersections.

Model results for track different track configurations are used to determine how
important the prevalence and placement of passing tracks is in preventing
delays. 

\subsection{Practical Application}

In 2014, one of the authors traveled by rail from Edinburgh, Scotland to London,
England.  The departure was the first train of the day with an expected travel
time of approximately four hours. Along the journey, and with numerous stops
remaining, the rail car the author was riding in suffered a breakdown.  After
exchanging passengers at a stop, the rail car doors would not close.  The rail
staff attempted for over twenty minutes to get the cars to close.  This resulted
in an immediate delay for all other stops and also delayed trains that would be
arriving later in the day.  With trains arriving and departing at major stops
every twenty to thirty minutes, this single breakdown had significant
repercussions for the remainder of the day. 

In addition to passengers still waiting at future stops on the way to London
needing to wait longer for the train to arrive, some passengers could not even
board the train when it did arrive.  With one car out of use, the train also
became overbooked.  The locals actually handled the perturbation relatively
well.  Some chose to stand on other cars when seats were not available.  Still
others chose to wait for the next train, which partially overbooks the later
trains.  The effortless adaptation of the locals implied that such delays were
common occurrences.

The challenge for the rail operators is trying to decrease the delays over time
in an effort to get back to normal operation.  This is a complex process with
many variables.  The number of trains and the frequency can each affect the
ability to overcome disruptions.  More trains on a given route could transport
more people, but a disruption would then have a much greater impact and the rail
operator would become more unlikely able to maintain a schedule.  Fewer trains
would mean less total passengers, but the ample time between trains would allow
delayed trains to speed up along a route and remain in travel behind a delayed
train. Rail operators need to model and simulate their networks in order to
account and plan for service disruptions.  Not only does this improve customer
satisfaction, but can it can improve financial performance as well.