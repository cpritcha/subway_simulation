\section{Results}

Summary statistics of the six different experiments are shown in table
~\ref{tab:experiment_results}. Each subway system simulation is run for two
hours (simulated time) 50 times to get information on the variability of
simulation runs. Passengers carried is a count and delay statistics are 
measured in minutes.

% In order to add results you must run doc/result_processing.ipynb in order to generate
% experiment_result_table.png
\begin{sidewaystable}
    \centering
    \caption{Experiment Comparison (50 replications each)}
    \begin{longtable}{llrrrr}
\toprule
                    & Name &  4 Trains No Delays &  4 Trains With Delays &  6 Trains No Delats &  6 Trains With Delays \\
\midrule
\endhead
\midrule
\multicolumn{3}{r}{{Continued on next page}} \\
\midrule
\endfoot

\bottomrule
\endlastfoot
 Passengers Carried & mean &            1,189.65 &              1,137.36 &            1,687.61 &              1,651.14 \\
                    & std &               56.44 &                 59.57 &               65.28 &                 72.03 \\
 Accumulated Load Unload Delays & mean &               16.14 &                 17.38 &               35.09 &                 33.69 \\
                    & std &                7.39 &                  9.64 &               10.62 &                 11.68 \\
 Average Load Unload Delay & mean &                0.08 &                  0.08 &                0.11 &                  0.11 \\
                    & std &                0.04 &                  0.05 &                0.04 &                  0.04 \\
 Average Load Unload Nonzero Delay & mean &                0.93 &                  1.02 &                1.04 &                  1.01 \\
                    & std &                0.31 &                  0.34 &                0.20 &                  0.19 \\
 Total Transit Delay & mean &                0.00 &                 30.11 &                0.00 &                 44.87 \\
                    & std &                0.00 &                  8.21 &                0.00 &                  8.14 \\
 Average Transit Delay & mean &                 nan &                  1.54 &                 nan &                  1.52 \\
                    & std &                 nan &                  0.23 &                 nan &                  0.19 \\
\end{longtable}
\label{tab:experiment_results}
\end{sidewaystable}

Experiment results match expectations. Introducing uniformly distributed zero
to three minute delays reduces the number of passengers that can be served in
two hours. Increasing the number of trains increases the number of passengers
served. Variable transit delays increase the variability of the system.

The schedule plots below show the trajectory of the trains on the Scarborough
line. Notice that trains heading in the same direction do not cross one
another. This means that there are no collisions in our simulation. Also notice
that the trajectories with delays have lower slopes than others for which
indicates a delay traveling from one station to another.